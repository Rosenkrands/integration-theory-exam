\documentclass{beamer}

\setbeamertemplate{section in toc}[sections numbered]
\setbeamertemplate{subsection in toc}[subsections numbered]

% american mathematical society
\usepackage{amsmath,amsthm,amssymb}

% drawing functionality
\usepackage{tikz}
\usetikzlibrary{matrix}

% equation numbering
\numberwithin{equation}{section}

% theorem types
\newtheorem{proposition}{Proposition}

% math operators
\DeclareMathOperator*{\argmin}{argmin}
\DeclareMathOperator*{\var}{Var}
\DeclareMathOperator*{\cov}{Cov}
\DeclareMathOperator{\VaR}{VaR}
\DeclareMathOperator{\cvar}{CVaR}

% additional mathematical fonts
\usepackage{mathrsfs}

% remove navigation bar
\beamertemplatenavigationsymbolsempty

% add slide numbers
\setbeamertemplate{footline}[frame number]

% title
\title{Financial Engineering}
\subtitle{Exam}
\author{Kasper Rosenkrands}
\institute{Aalborg University}
\date{S20}

% toc at each section
\AtBeginSection[]
{
  \begin{frame}
    \frametitle{Table of Contents}
    \tableofcontents[currentsection, hideothersubsections]
  \end{frame}
}

% definition of comment function itself
\newcommand{\comment}[1]{
    \begin{center}
        \colorbox{yellow}{
            \textsf{
                \textbf{#1}
            }
        }
    \end{center}
}
\newcommand{\task}[1]{
    \begin{center}
        \colorbox{red}{
            \textsf{
                \textbf{#1}
            }
        }
    \end{center}
}

\begin{document}

\frame{\titlepage}

\begin{frame}
\frametitle{Table of Contents}
\tableofcontents[hideallsubsections]
\end{frame}

\section{Lebesgue integration theory}

\subsection{Monotone Convergence Theorem}

\begin{frame}\frametitle{{\normalsize \secname} \\ {\large \subsecname}}
    \begin{theorem}[Monotone Convergence Thereom]
        If $(f_n)$ is a monotone increasing sequence of functions in $M^+(X,m)$ which converges to $f$, then
        \begin{align}\label{eq:b4.6}
            \int f \,d\mu = \lim \int f_n  \,d\mu. 
        \end{align}
    \end{theorem}
\end{frame}

\subsection{Proof of Monotone Convergence Theorem}

\begin{frame}\frametitle{{\normalsize \secname} \\ {\large \subsecname}}
    The strategy of the proof is to first show that
    \begin{align}
        \lim \int f_n \, d\mu \leq \int f \, d\mu,
    \end{align}
    then afterwards to show that also
    \begin{align}
        \lim \int f_n \, d\mu \geq \int f \, d\mu,
    \end{align}
    in order to conclude that
    \begin{align}
        \lim \int f_n \, d\mu = \int f \, d\mu
    \end{align}
\end{frame}

\begin{frame}\frametitle{{\normalsize \secname} \\ {\large \subsecname}}
    According to Corollary 2.10
    \begin{corollary}[2.10]
        If $(f_n)$ is a sequence in $M(X,m)$ which converges to $f$ on $X$, the $f$ is in $M(X,m)$.
    \end{corollary}
    the function $f$ is measurable.
\end{frame}

\begin{frame}\frametitle{{\normalsize \secname} \\ {\large \subsecname}}
    From Lemma 4.5(1.)
    \begingroup
    \small
    \begin{lemma}
        \begin{enumerate}
            \item If $f$ and $g$ belong to $M^+(X,m)$ and $f \leq g$, then
            \begin{align}
                \int f \, d\mu \leq \int g \, d\mu.
            \end{align}
            \item If $f$ belongs to $M^+(X,m)$, if $E$, $F$ belong to $m$, and if $E\subseteq F$, then
            \begin{align}
                \int_E f \, d\mu \leq \int_F f \, d\mu.
            \end{align}
        \end{enumerate}
    \end{lemma}
    \endgroup
    we have that
    \begin{align}
        \int f_n \, d\mu \leq \int f_{n + 1} \, d\mu \leq \int f \, d\mu, \quad \forall \, n \in \mathbb{N}.
    \end{align}
\end{frame}

\begin{frame}\frametitle{{\normalsize \secname} \\ {\large \subsecname}}
    Therefore we must also have that
    \begin{align}
        \lim \int f_n \, d\mu \leq \int f \, d\mu.
    \end{align}
    So this was the first step of our strategy, now we proceed to the second step.
\end{frame}

\begin{frame}\frametitle{{\normalsize \secname} \\ {\large \subsecname}}
    Let $\alpha \in \mathbb{R}$ be such that $0 < \alpha < 1$ and let $\varphi$ be a simple measurable function such that $0 \leq \varphi \leq f$.
    
    Let
    \begin{align}
        A_n = \left\{x \in X \, : \, f_n(x) \geq \alpha \varphi(x) \right\},
    \end{align}
    such that
    \begin{enumerate}
        \item $A_n \in m$
        \item $A_n \subseteq A_{n + 1}$
        \item $X = \bigcup A_n$
    \end{enumerate}
\end{frame}

\begin{frame}\frametitle{{\normalsize \secname} \\ {\large \subsecname}}
    According to Lemma 4.5
    \begingroup
    \small
    \begin{lemma}
        \begin{enumerate}
            \item If $f$ and $g$ belong to $M^+(X,m)$ and $f \leq g$, then
            \begin{align}
                \int f \, d\mu \leq \int g \, d\mu.
            \end{align}
            \item If $f$ belongs to $M^+(X,m)$, if $E$, $F$ belong to $m$, and if $E\subseteq F$, then
            \begin{align}
                \int_E f \, d\mu \leq \int_F f \, d\mu.
            \end{align}
        \end{enumerate}
    \end{lemma}
    \endgroup
    it must be that
    \begin{align}\label{eq:bartle_4.7}
        \int_{A_n} \alpha \varphi \, d\mu \leq \int_{A_n} f_n \, d\mu \leq \int f_n \, d\mu.
    \end{align}
\end{frame}

\begin{frame}\frametitle{{\normalsize \secname} \\ {\large \subsecname}}
    Since the sequence $A$ is monotone increasing and has union $X$, it follows from Lemma 4.3(2.) and Lemma 3.4(1.),
    \begingroup
    \footnotesize
    \begin{columns}
        \begin{column}{0.48\textwidth}
            \begin{lemma}[4.3]
                \begin{enumerate}
                    \item If $\varphi$ and $\psi$ are simple functions in $M^+(X,m)$ and $c \geq 0$, then
                    \begin{align}
                        \int c \varphi \, d\mu &= c \int \varphi \, d\mu, \\
                        \int \left( \varphi + \psi \right) \, d\mu &= \int \varphi \, d\mu + \int \psi \, d\mu.
                    \end{align}
                    \item If $\lambda$ is defined for $E$ in $m$ by
                    \begin{align}
                        \lambda(E) = \int \varphi \chi_E \, d\mu,
                    \end{align}
                    then $\lambda$ is a measure on $m$.
                \end{enumerate}
            \end{lemma}
        \end{column}
        \begin{column}{0.48\textwidth}
            \begin{lemma}[3.4]
                Let $\mu$ be a measure defined on a $\sigma$-algebra $m$.
                \begin{enumerate}
                    \item If $(E_n)$ is an increasing sequence in $m$, then
                    \begin{align}
                        \mu\left(\bigcup_{n = 1}^\infty E_n\right) = \lim \mu\left(E_n\right).
                    \end{align}
                    \item If $(F_n)$ is a decreasing sequence in $m$ and if $\mu\left(F_1\right) < +\infty$, then
                    \begin{align}
                        \mu\left(\bigcap_{n = 1}^\infty F_n\right) = \lim \mu\left(F_n\right).
                    \end{align}
                \end{enumerate}
            \end{lemma}
        \end{column}
    \end{columns}
    \endgroup
\end{frame}

\begin{frame}\frametitle{{\normalsize \secname} \\ {\large \subsecname}}
    that
    \begin{align}
        \int \varphi \, d\mu = \lim \int_{A_n} \varphi \, d\mu.
    \end{align}
    Taking the limit as $n$ tends to infinity in \eqref{eq:bartle_4.7} therefore gives that
    \begin{align}
        \alpha \int \varphi \, d\mu \leq \lim \int f_n \, d\mu.
    \end{align}
    Since this holds for all $0 < \alpha < 1$, by taking the limit as $\alpha$ tends to 1 we obtain
    \begin{align}
        \int \varphi \, d\mu \leq \lim \int f_n \, d\mu.
    \end{align}
\end{frame}

\begin{frame}\frametitle{{\normalsize \secname} \\ {\large \subsecname}}
    As $\varphi$ is any simple function in $M^+$ such that $0 \leq \varphi \leq f$, we can conclude that
    \begin{align}
        \int f \, d\mu = \sup_{\varphi} \int \varphi \, d\mu \leq \lim \int f_n \, d\mu,
    \end{align}
    which concludes the proof. \hfill $\square$
\end{frame}

\section{$\mathbf{L^p}$ spaces}

\begin{frame}\frametitle{{\normalsize \secname} \\ {\large \subsecname}}
\end{frame}

\section{Decomposition of measures}

\begin{frame}\frametitle{{\normalsize \secname} \\ {\large \subsecname}}
\end{frame}

\section{Generation of measures and product measures}

\begin{frame}\frametitle{{\normalsize \secname} \\ {\large \subsecname}}
\end{frame}

\section{Approximation by nice functions}

\begin{frame}\frametitle{{\normalsize \secname} \\ {\large \subsecname}}
\end{frame}

\section{Fourier transform}

\begin{frame}\frametitle{{\normalsize \secname} \\ {\large \subsecname}}
\end{frame}

\end{document}