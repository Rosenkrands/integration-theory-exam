\documentclass{article}
\usepackage{amsmath,amsthm,amssymb}
\newtheorem{theorem}{Theorem}[section]
\numberwithin{equation}{section}

\begin{document}

\thispagestyle{empty}
\begin{center}
{\huge\textbf{Integration Theory}}\\[2mm]
{\Large Kasper Rosenkrands}\\[2cm]
{\large MATØK6}\\[2mm]
{\large Spring 2020}
\end{center}

\newpage

\pagenumbering{arabic}

\section{Lebesque Integration Theory}
(Chapter 2-5 of Bartle's book, the most important results are: monotone convergence theorem, Fatou's lemma and Lebesgue dominated convergence theorem)

4.6 - Monotone Convergence Thereom

\begin{theorem}[Monotone Convergence Thereom]
    If $(f_n)$ is a monotone increasing sequence of functions in $M^+(X,m)$ which converges to $f$, then
    \begin{align}\label{eq:b4.6}
        \int f \,d\mu = \lim \int f_n  \,d\mu. 
    \end{align}
\end{theorem}

4.8 - Fatou's Lemma

\begin{theorem}[Fatou's Lemma]
    If $(f_n)$ belongs to $M^+(X,m)$, then
    \begin{align}\label{eq:b4.8}
        \int (\liminf f_n) \, d\mu \leq \liminf \int f_n \, d\mu.
    \end{align}
\end{theorem}

5.6 - Lebesgue Dominated Convergence Theorem

\begin{theorem}[Lebesgue Dominated Convergence Theorem]
    Let $(f_n)$ be a sequence of integrable function which converges almost everywhere to a real-valued measureable function $f$.
    If there exists an integrable function $g$ such that $|f_n|\leq g$ for all $n$, then $f$ is integrable and
    \begin{align}\label{eq:b5.4}
        \int f \, d\mu = \lim \int f_n \, d\mu.
    \end{align}
\end{theorem}


\newpage

\section{$\mathbf{L^p}$ Spaces}
(Chapter 6 of Bartle's book, the most important results are: Hölder's inequality, Minkowski's inequality and Riesz-Fischer Theorem)

6.9 - Hölder's Inequality

\begin{theorem}[Hölder's Inequality]
    Let $f\in L_p$ and $g \in L_q$ where $p>1$ and $(1/p) + (1/q) = 1$.
    Then $fg\in L_1$ and  $\| fg \|_1 \leq \|f\|_p \|g\|_q$.
\end{theorem}

6.11 - Minkowski's Inequality

\begin{theorem}[Minkowski's Inequality]
    If $f$ and $h$ belong to $L_p$, $p \geq 1$, then $f + h$ belongs to $L_p$ and
    \begin{align}\label{eq:b6.6}
        \| f + h \|_p \leq \| f \|_p + \| h \|_p.
    \end{align}
\end{theorem}

6.14 - Completeness Theorem (Riesz-Fischer Theorem)

\begin{theorem}[Completeness Theorem (Riesz-Fischer Theorem)]
    If $1\leq p < \infty$, then the space $L_p$ is a complete normed linear space under the norm
    \begin{align}\label{eq:b-under_th6.14}
        \| f \|_p = \left\{ \int |f|^p \, d\mu \right\}^{1/p}.
    \end{align}
\end{theorem}

\newpage

\section{Decomposition of Measures}
(Chapter 8 of Bartle's book, the most important results are: Radon-Nikodym theorem, Lebesgue decomposition theorem and Riesz representation theorem)

8.9 - Radon-Nikodým Theorem

\begin{theorem}[Radon-Nikodým Theorem]
    Let $\lambda$ and $\mu$ be $\sigma$-finite measures defined on $m$ and suppose that $\lambda$ is absolutely continuous with respect to $\mu$.
    Then there exists a function $f$ in $M^+(X,m)$ such that
    \begin{align}\label{eq:b8.6}
        \lambda(E) = \int_E f \, d\mu, \quad E \in m.
    \end{align}
    Moreover, the functino $f$ is uniquely determined $\mu$-almost everywhere.
\end{theorem}

8.11 - Lebesgue Decomposition Theorem

\begin{theorem}[Lebesgue Decomposition Theorem]
    Let $\lambda$ and $\mu$ be $sigma$-finite measures defined on a $sigma$-algebra $m$.
    Then there exists a measure $\lambda_1$ which is singular with respect to $\mu$ and a measure $\lambda_2$ which is absolutely continuous with respect to $\mu$ such that $\lambda = \lambda_1 + \lambda_2$.
    Moreover, the measures $\lambda_1$ and $\lambda_2$ are unique.
\end{theorem}

8.14 - Riesz Representation Theorem

\begin{theorem}[Riesz Representation Theorem]
    If $(X, m, \mu)$ is a $sigma$-finite measure space and $G$ is a bounded linear functional on $L_1(X, m, \mu)$, then there exists a $g$ in $L_\infty(X, m, \mu)$ such that
    \begin{align}\label{eq:b8.10}
        G(f) = \int fg \, d\mu
    \end{align}
    holds for all $f$ in $L_1$.
    Moreover, $\|G\| = \|g\|_\infty$ and $g \geq 0$ if $G$ is a positive linear functional. 
\end{theorem}

\newpage

\section{Generation of Measures and Product Measures}

(Chapter 9 and 10 of Bartle's book, the most important results are: Carathéodory extension theorem, Hahn extension theorem, Product measure theorem, Fubini's theorem and Tonelli's theorem)

9.7 - Carathéodory Extension Theorem

\begin{theorem}[Carathéodory Extension Theorem]
    the collection $A^*$ of all $\mu^*$-measureable sets is a $\sigma$-algebra containing $A$.
    Moreover, if $(E_n)$ is a disjoint sequence in $A^*$, then
    \begin{align}\label{eq:b9.7}
        \mu^*\left(\bigcup_{n = 1}^\infty E_n \right)
        = \sum_{n = 1}^\infty \mu^*(E_n).
    \end{align}
\end{theorem}

9.8 - Hahn Extension Theorem

\begin{theorem}[Hahn Extension Theorem]
    Suppose that $\mu$ is a $\sigma$-finite measure on an algebra $A$.
    Then there exists a unique extension of $\mu$ to a measure on $A^*$.
\end{theorem}

10.4 - Product Measure Thereom

\begin{theorem}[Product Measure Thereom]
    If $(X, m, \mu)$ and $(Y, n, \nu)$ are measure spaces, then there exists a measure $\pi$ defined $Z_0 = m \times n$ such that
    \begin{align}\label{eq:b10.1}
        \pi(A \times B) = \mu(A)\nu(B)
    \end{align}
    for all $A \in m$ and $B \in n$.
    If these measure spaces are $\sigma$-fintie, then there is a unique measure $\pi$ with property \eqref{eq:b10.1}.
\end{theorem}

10.9 - Tonelli's Theorem

\begin{theorem}[Tonelli's Theorem]
    Let $(X, m, \mu)$ and $(Y, n, \nu)$ be a $\sigma$-finite measure space and let $F$ be a nonnegative measureable function on $Z = X \times Y$ to $\overline{\mathbb{R}}$.
    Then the functions defined on $X$ and $Y$ by
    \begin{align}\label{eq:b10.4}
        f(x) = \int_Y F_x \, d\nu, \quad
        g(y) = \int_X F^y \, d\mu,
    \end{align}
    are measureable and
    \begin{align}\label{eq:b10.5}
        \int_X f \, d\mu = \int_Z F \, d\pi = \int_Y g \, d\nu.
    \end{align}
    In other symbols,
    \begin{align}\label{eq:b10.6}
        \int_X\left(\int_Y F \, d\nu \right) \, d\mu =
        \int_Z F \, d\pi =
        \int_Y\left(\int_X F \, d\mu \right) \, d\nu.
    \end{align}
\end{theorem}

10.10 - Fubini's Theorem

\begin{theorem}[Fubini's Theorem]
    Let $(X, m, \mu)$ and $(Y, n, \nu)$ be $\sigma$-finite spaces and let the measure $\pi$ on $Z_0 = m \times n$ be the product measure of $\mu$ and $\nu$.
    If the function $F$ on $Z = X \times Y$ to $\mathbb{R}$ is integrable with respect to $\pi$, then the extended real-valued functions defined almost everywhere by
    \begin{align}\label{eq:b10.8}
        f(x) = \int_Y F_x \, d\nu, \quad
        g(y) = \int_X F^y \, d\mu
    \end{align}
    have finite integrals and
    \begin{align}\label{eq:b10.9}
        \int_x f \, d\mu = \int_Z F \, d\pi = \int_Y g \, d\nu. 
    \end{align}
    In other symbols,
    \begin{align}\label{eq:b10.10}
        \int_X \left[\int_Y F \, d\nu\right] \, d\mu =
        \int_Z F \, d\pi =
        \int_Y \left[\int_X F \, d\mu\right] \, d\nu.
    \end{align}
\end{theorem}

\newpage

\section{Approximation by Nice Functions}
(Lecture 11-12, the main results are: density in $L^p$ of continuous function with constant support, density in $L^p$ of smooth functions with constant support using the approximate identity).

\newpage

\section{Fourier Transform}
\end{document}