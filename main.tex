\documentclass{article}
\usepackage{amsmath,amsthm,amssymb}
\newtheorem{theorem}{Theorem}[section]
\numberwithin{equation}{section}

\begin{document}

\thispagestyle{empty}
\begin{center}
{\huge\textbf{Integration Theory}}\\[2mm]
{\Large Kasper Rosenkrands}\\[2cm]
{\large MATØK6}\\[2mm]
{\large Spring 2020}
\end{center}

\newpage

\pagenumbering{arabic}

\section{Lebesque Integration Theory}
(Chapter 2-5 of Bartle's book, the most important results are: monotone convergence theorem, Fatou's lemma and Lebesgue dominated convergence theorem)

4.6 - Monotone Convergence Thereom

\begin{theorem}[Monotone Convergence Thereom]
    If $(f_n)$ is a monotone increasing sequence of functions in $M^+(X,m)$ which converges to $f$, then
    \begin{align}\label{eq:b4.6}
        \int f \,d\mu = \lim \int f_n  \,d\mu. 
    \end{align}
\end{theorem}

4.8 - Fatou's Lemma

\begin{theorem}[Fatou's Lemma]
    If $(f_n)$ belongs to $M^+(X,m)$, then
    \begin{align}\label{eq:b4.8}
        \int (\liminf f_n) \, d\mu \leq \liminf \int f_n \, d\mu.
    \end{align}
\end{theorem}

5.6 - Lebesgue Dominated Convergence Theorem

\begin{theorem}[Lebesgue Dominated Convergence Theorem]
    Let $(f_n)$ be a sequence of integrable function which converges almost everywhere to a real-valued measureable function $f$.
    If there exists an integrable function $g$ such that $|f_n|\leq g$ for all $n$, then $f$ is integrable and
    \begin{align}\label{eq:b5.4}
        \int f \, d\mu = \lim \int f_n \, d\mu.
    \end{align}
\end{theorem}


\newpage

\section{$\mathbf{L^p}$ Spaces}
(Chapter 6 of Bartle's book, the most important results are: Hölder's inequality, Minkowski's inequality and Riesz-Fischer Theorem)

6.9 - Hölder's Inequality

\begin{theorem}[Hölder's Inequality]
    Let $f\in L_p$ and $g \in L_q$ where $p>1$ and $(1/p) + (1/q) = 1$.
    Then $fg\in L_1$ and  $\| fg \|_1 \leq \|f\|_p \|g\|_q$.
\end{theorem}

6.11 - Minkowski's Inequality

\begin{theorem}[Minkowski's Inequality]
    If $f$ and $h$ belong to $L_p$, $p \geq 1$, then $f + h$ belongs to $L_p$ and
    \begin{align}\label{eq:b6.6}
        \| f + h \|_p \leq \| f \|_p + \| h \|_p.
    \end{align}
\end{theorem}

6.14 - Completeness Theorem (Riesz-Fischer Theorem)

\begin{theorem}[Completeness Theorem (Riesz-Fischer Theorem)]
    If $1\leq p < \infty$, then the space $L_p$ is a complete normed linear space under the norm
    \begin{align}\label{eq:b-under_th6.14}
        \| f \|_p = \left\{ \int |f|^p \, d\mu \right\}^{1/p}.
    \end{align}
\end{theorem}

\newpage

\section{Decomposition of Measures}
(Chapter 8 of Bartle's book, the most important results are: Radon-Nikodym theorem, Lebesgue decomposition theorem and Riesz representation theorem)

8.9 - Radon-Nikodým Theorem

8.11 - Lebesgue Decomposition Theorem

8.14 - Riesz Representation Theorem

\newpage

\section{Generation of Measures and Product Measures}

(Chapter 9 and 10 of Bartle's book, the most important results are: Carathéodory extension theorem, Hahn extension theorem, Product measure theorem, Fubini's theorem and Tonelli's theorem)

9.7 - Carathéodory Extension Theorem

9.8 - Hahn Extension Theorem

10.4 - Product Measure Thereom

10.9 - Tonelli's Theorem

10.10 - Fubini's Theorem

\newpage

\section{TBA}

\newpage

\section{TBA}
\end{document}